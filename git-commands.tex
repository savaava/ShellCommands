\subsection*{Introduzione}
\addcontentsline{toc}{subsection}{Introduzione}
Questa sezione presenta un elenco di comandi git per gestire la propria repository in locale e per interagire con la repository in remoto sul server GitHub.\vspace{.3cm}\\
Si utilizza la shell \texttt{GitBash} perche è più simile a quella di \texttt{Linux}, in alternativa si può usare \texttt{PowerShell} o \texttt{cmd} di \texttt{Windows} che invece richiedono i comandi nativi di \texttt{Windows}.\vspace{.3cm}\\
E' possibile avere una repository git su un qualsiasi path (sul Desktop).\\
Una cartella si definisce repository git in locale al dispositivo quando è presente la sottocartella nascosta \texttt{.git}, che consente di usare i relativi comandi.\\
La repository in locale è diversa da quella che si ha su un server remoto ad esempio su GitHub, ed è possibile collegarle in modo che le modifiche (aggiunte o rimozioni di file) effettuate nella repository locale vengano caricate sul server, ossia sulla rispettiva repository remota.


\subsection{Inizializzazione della repository}
Se si desidera collegare la repository locale ad una remota, allora assicurarsi di aver creato prima la repository in remoto su GitHub (\url{https://github.com/new}) e di averne prelevato l'\texttt{\color{blue!70!black}"URL-remoto"}, che può essere HTTPS o SSH.

\subsubsection{git clone}
\begin{lstlisting}[language=bash]
	git clone "URL-remoto"
\end{lstlisting}
Clona la repository remota specificata dall'URL nel percorso in cui è stato eseguito il comando, creando una nuova cartella con lo stesso nome del remoto e aggiornata con gli stessi dati.
Crea quindi una repository locale (\texttt{.git}), collegandola direttamente a quella remota.\\
In particolare si crea un branch \texttt{main} in locale che è collegato a quello remoto \texttt{remotes/origin/main}.\\
E' un comando molto più comodo e veloce per la creazione e il collegamento della repository, ma in alternativa si possono usare i comandi \texttt{git init} e \texttt{git remote}.

\subsubsection{git init}
\begin{lstlisting}[language=bash]
	git init
\end{lstlisting}
Inizializza la cartella in cui si esegue il comando come una nuova repository in locale (\texttt{.git}).\\
E' possibile lavorare solo in locale (tramite comandi come \texttt{git add}, \texttt{git commit}, \dots), in quanto la repository locale non è stata collegata a nessuna repository in un server remoto.

\subsubsection{git remote}
\begin{lstlisting}[language=bash]
	git remote add origin "URL-remoto"
\end{lstlisting}
Collega la repository locale creata con \texttt{git init} con una remota già esistente.\\
In seguito a questo passaggio si ha che la repository in locale è collegata ad una nuova in remoto.\\

\begin{lstlisting}[language=bash]
	git remote set-url origin "URL-remoto"
\end{lstlisting}
Collega la repository locale ad un altro repository remoto, modificando l'URL del remoto esistente.\\
Questo può servire quando si vuole collegare il repository locale a quello remoto con l'URL SSH, sostituendolo a quello HTTPS, o viceversa.\\
Oppure può servire per sostituire l'URL vecchio del repository remoto con uno nuovo, perchè è stato eventualmente cambiato.\\

\begin{lstlisting}[language=bash]
	git remote -v
\end{lstlisting}
Mostra l'elenco dei repository remoti associati al repository locale, insieme agli URL per \texttt{fetch} e \texttt{push}.
\begin{lstlisting}[language=bash]
	# output:
	origin  https://github.com/savaava/ShellCommands.git (fetch)
	origin  https://github.com/savaava/ShellCommands.git (push)
\end{lstlisting}
Il nome dei repository remoti è di default \texttt{origin} e viene mostrato il repository remoto, tramite il suo URL, da cui si prelevano le commit (fetch) e quello su cui si caricano le commit (push).


\subsection{Aggiunta/caricamento}
\subsection{Ripristinare commit}
\subsection{Branch}
\subsection{Stash}
\subsection{Altri}
\subsection{Configurazione}