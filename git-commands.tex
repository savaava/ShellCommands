\subsection*{Introduzione}
\addcontentsline{toc}{subsection}{Introduzione}
Questa sezione presenta un elenco di comandi git per gestire la propria repository in locale e per interagire con la repository in remoto sul server GitHub.\vspace{.3cm}\\
Si utilizza la shell \texttt{GitBash} perche è più simile a quella di \texttt{Linux}, in alternativa si può usare \texttt{PowerShell} o \texttt{cmd} di \texttt{Windows} che invece richiedono i comandi nativi di \texttt{Windows}.


\subsection{Inizializzazione della repository}
\begin{lstlisting}[language=bash]
	git init
\end{lstlisting}
Inizializza la cartella in cui si esegue il comando come una nuova repository Git in locale, creando la sottocartella nascosta \texttt{.git} che garantisce l'utilizzo dei comandi git.\\
E' possibile lavorare solo in locale (tramite comandi come \texttt{git add}, \texttt{git commit}, \dots), in quanto la repository locale non è stata collegata a nessuna repository in un server remoto, che può essere GitHub.\\

\begin{lstlisting}[language=bash]
	git remote add origin "URL-remoto"
\end{lstlisting}
Collega la repository locale creata con \texttt{git init} con una remota già esistente.\\ Assicurarsi di aver creato correttamente la repository in remoto su GitHub e di averne prelevato l'\texttt{\color{blue!70!black}"URL-remoto"}.
In seguito a questo passaggio si ha che la repository in locale è collegata ad una in remoto, e di default vi sono due branch legati:
\begin{itemize}[noitemsep, topsep=0pt]
	\item \texttt{main} -> nella repository locale
	\item \texttt{remotes/origin/main} -> nella repository remota
\end{itemize}


\subsection{Aggiunta/caricamento}
\subsection{Ripristinare commit}
\subsection{Branch}
\subsection{Stash}
\subsection{Altri}
\subsection{Configurazione}