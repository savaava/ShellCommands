\documentclass[12pt, a4paper]{article}

% ******************************** PACKAGES ********************************
\usepackage[top=2cm, bottom=2cm, left=1.5cm, right=1.5cm]{geometry}
\usepackage{listings}
\usepackage{xcolor}
\usepackage{enumitem}
\usepackage{fancyhdr}
\usepackage[colorlinks=true]{hyperref}
\pagestyle{fancy}

% ******************************** SETUP ********************************
\lstset{
	basicstyle=\ttfamily\color{black}, % Testo principale in nero
	commentstyle=\color{green!50!black}, % Commenti in verde scuro
	stringstyle=\color{blue!70!black}, % Stringhe in blu scuro
	backgroundcolor=\color{gray!15}, % Sfondo grigio chiaro
	frame=single, % Cornice attorno al codice
	numbers=left, % Numeri di linea a sinistra
	numberstyle=\small\color{gray!70}, % Numeri in grigio tenue
	%breaklines=true, % Spezza le righe troppo lunghe
	showstringspaces=false,
	tabsize=1, % Dimensione del tab	
	keywordstyle=\color{red!70!black}\bfseries, % Parole chiave in rosso scuro e grassetto
}

\newcommand{\disablelinkcolor}{%
	\hypersetup{linkcolor=black}%
}
\newcommand{\enablelinkcolor}{
	\hypersetup{
		%hidelinks,
		linkcolor=blue
		%linkbordercolor=blue,
		%pdfborderstyle={/S/U/W 1} % Bordo sottolineato (stile underline)
	}
}

% ******************************** MAIN PAGE INFO ********************************
\title{Comandi in Shell}
\author{Andrea Savastano}
\date{\today}

% ******************************** DOCUMENTO ********************************
\begin{document}	
	% Configurazione dell'intestazione
	\fancyhf{} % Resetta intestazioni e piè di pagina
	\fancyhead[L]{\nouppercase{\leftmark}} % Testata sinistra: sezione corrente
	\fancyhead[R]{page \thepage} % Testata destra: numero di pagina
	
	\maketitle
	\disablelinkcolor
	\tableofcontents
	\newpage
	
	\section{Commandi git}
	\subsection*{Introduzione}
\addcontentsline{toc}{subsection}{Introduzione}
Questa sezione presenta un elenco di comandi git per gestire la propria repository in locale e per interagire con la repository in remoto sul server GitHub.\vspace{.3cm}\\
Si utilizza la shell \texttt{GitBash} perche è più simile a quella di \texttt{Linux}, in alternativa si può usare \texttt{PowerShell} o \texttt{cmd} di \texttt{Windows} che invece richiedono i comandi nativi di \texttt{Windows}.\vspace{.3cm}\\
E' possibile avere una repository git su un qualsiasi path (sul Desktop).\\
Una cartella si definisce repository git in locale al dispositivo quando è presente la sottocartella nascosta \texttt{.git}, che consente di usare i relativi comandi.\\
La repository in locale è diversa da quella che si ha su un server remoto ad esempio su GitHub, ed è possibile collegarle in modo che le modifiche (aggiunte o rimozioni di file) effettuate nella repository locale vengano caricate sul server, ossia sulla rispettiva repository remota.


\subsection{Inizializzazione della repository}
Se si desidera collegare la repository locale ad una remota, allora assicurarsi di aver creato prima la repository in remoto su GitHub (\url{https://github.com/new}) e di averne prelevato l'\texttt{\color{blue!70!black}"URL-remoto"}, che può essere HTTPS o SSH.

\subsubsection{git clone}
\begin{lstlisting}[language=bash]
	git clone "URL-remoto"
\end{lstlisting}
Clona la repository remota specificata dall'URL nel percorso in cui è stato eseguito il comando, creando una nuova cartella con lo stesso nome del remoto e aggiornata con gli stessi dati.
Crea quindi una repository locale (\texttt{.git}), collegandola direttamente a quella remota.\\
In particolare si crea un branch \texttt{main} in locale che è collegato a quello remoto \texttt{remotes/origin/main}.\\
E' un comando molto più comodo e veloce per la creazione e il collegamento della repository, ma in alternativa si possono usare i comandi \texttt{git init} e \texttt{git remote}.

\subsubsection{git init}
\begin{lstlisting}[language=bash]
	git init
\end{lstlisting}
Inizializza la cartella in cui si esegue il comando come una nuova repository in locale (\texttt{.git}).\\
E' possibile lavorare solo in locale (tramite comandi come \texttt{git add}, \texttt{git commit}, \dots), in quanto la repository locale non è stata collegata a nessuna repository in un server remoto.

\subsubsection{git remote}
\begin{lstlisting}[language=bash]
	git remote add origin "URL-remoto"
\end{lstlisting}
Collega la repository locale creata con \texttt{git init} con una remota già esistente.\\
In seguito a questo passaggio si ha che la repository in locale è collegata ad una nuova in remoto.\\

\begin{lstlisting}[language=bash]
	git remote set-url origin "URL-remoto"
\end{lstlisting}
Collega la repository locale ad un altro repository remoto, modificando l'URL del remoto esistente.\\
Questo può servire quando si vuole collegare il repository locale a quello remoto con l'URL SSH, sostituendolo a quello HTTPS, o viceversa.\\
Oppure può servire per sostituire l'URL vecchio del repository remoto con uno nuovo, perchè è stato eventualmente cambiato.\\

\begin{lstlisting}[language=bash]
	git remote -v
\end{lstlisting}
Mostra l'elenco dei repository remoti associati al repository locale, insieme agli URL per \texttt{fetch} e \texttt{push}.
\begin{lstlisting}[language=bash]
	# output:
	origin  https://github.com/savaava/ShellCommands.git (fetch)
	origin  https://github.com/savaava/ShellCommands.git (push)
\end{lstlisting}
Il nome dei repository remoti è di default \texttt{origin} e viene mostrato il repository remoto, tramite il suo URL, da cui si prelevano le commit (fetch) e quello su cui si caricano le commit (push).


\subsection{Aggiunta/caricamento}
Quando si aggiorna la repository locale, aggiungendo/eliminando/modificando un file, è possibile effettuare una commit, la quale carica le modifiche in locale. Per aggiornare anche la repository remota a cui è collegata quella locale si deve effettuare una push della commit.\\
Pertanto, dopo aver aggiornato la repository locale, i passaggi da seguire sono 3:
\begin{center}
\begin{minipage}{0.5\textwidth}
\begin{tcolorbox}[left=0cm, colframe=red!50!gray]
\begin{enumerate}[noitemsep, topsep=0pt]
	\item \texttt{git add}
	\item \texttt{git commit}
	\item \texttt{git push}
\end{enumerate}
\end{tcolorbox}
\end{minipage}
\end{center}
E' importante monitorare sempre lo stato della repository con \texttt{git status}, per tenere traccia dei cambiamenti in corso d'opera.

\subsubsection{git status}
\begin{lstlisting}[language=bash]
	git status
\end{lstlisting}
Mostra lo stato della repository locale, specificando i file che sono stati aggiunti, modificati o eliminati.\\
Il comando in esame fornisce informazioni anche sui file che la repo sta monitorando:
\begin{itemize}[noitemsep, topsep=3pt]
	\item I file \textbf{\color{red}Untracked} non sono ancora monitorati da Git (non aggiunti con \texttt{git add}), quindi sono i file aggiunti e non ancora caricati nella repo per la prima volta;
	\item I file \textbf{Tracked} sono già sotto il controllo di Git, quindi sono i file già caricati precedentemente nella repository, e possono essere \textbf{Unmodified}, \textbf{\color{red}Modified} o \textbf{\color{red}Deleted}.
\end{itemize}
I file Untracked, Modified e Deleted sono colorati in rosso quando non sono stati ancora aggiunti alla Staging area con \texttt{git add}. Una volta aggiunti saranno verdi.

\subsubsection{git add}
a
\subsubsection{git commit}
\subsubsection{git push}
\subsubsection{git fetch}
\subsubsection{git pull}


\subsection{Ripristinare commit}



\subsection{Branch}



\subsection{Stash}



\subsection{Altri}



\subsection{Configurazione}



	
	\newpage
	\section{Comandi ssh}
	\subsection*{Introduzione}
\addcontentsline{toc}{subsection}{Introduzione}
Questa sezione spiega come gestire una connessione SSH tra il dispositivo client e quello server utilizzando OpenSSH (Open Secure Shell), che consente di eseguire operazioni remote su altri computer attraverso una rete in modo sicuro sulla porta 22.\vspace{.3cm}\\
Si utilizza \texttt{PowerShell} di \texttt{Windows} e \texttt{GitBash} eseguiti come amministratore, in alternativa è possibile utilizzare \texttt{cmd} sempre come amministratore.

\subsection{Inizializzazione dei dispositivi}
\subsection{Apertura/Chiusura della connessione}
\subsection{Operazioni remote}

	
	\newpage
	\section{Comandi bash}
	\input{bash-commands.tex}
	
\end{document}
